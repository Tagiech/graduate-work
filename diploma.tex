\documentclass[a4paper]{article}
\usepackage[12pt]{extsizes}
\usepackage[left=2cm,right=2cm,top=2.5cm,bottom=2.5cm,bindingoffset=0cm]{geometry}
\usepackage[utf8]{inputenc}
\usepackage[russian]{babel}
\usepackage{amsmath,amssymb}
\usepackage{mathtext}
\usepackage[T2A]{fontenc}
\linespread{1.2}
\usepackage{tikz}
\usepackage{pgfplots}
\usepackage{graphicx}
\graphicspath{{pictures/}} 
\setcounter{secnumdepth}{4}



\begin{document}

\begin{titlepage}
\begin{center}
    {\bf
    Санкт-Петербургский государственный университет}
    
    \vspace{3cm}
    
    {\bf ТАГИЕВ Артур Мехман оглы 
    
    \, \\ 
    
    Выпускная квалификационная работа 
    
    \textit{Исследование динамики ГНСС-пункта \\ на антарктической станции Восток}}
    
    \vspace{2cm}
    Уровень образования: специалитет
    
    Направление 03.05.01 <<Астрономия>>
    
    Основная образовательная программа СМ.5012.2016 <<Астрономия>>
    
    Профиль <<Астрономия>>
\end{center}

\vspace{2cm}
\begin{flushright}
    \begin{tabular}{ll}
         & Научный руководитель: \\
         & доцент кафедры астрономии, \\
         & кандидат физико-математических наук \\
         & Петров Сергей Дмитриевич
         & \\
         & Рецензент:\\
         & научный сотрудник ГАО РАН \\
         & Щербакова Наталия Васильевна
    \end{tabular}
\end{flushright}

     \vfill
\begin{center}
    Санкт-Петербург 2022
\end{center} 
\end{titlepage}

\tableofcontents
\newpage

 \section{Введение}
 
 Данная работа посвящена континенту, открытие которого произошло последним -- Антарктиде. Открыли её 28 января 1820 года, и до сих пор Антарктида является самым неизученным континентом. Изучение этого материка осложнятся суровыми условиями: низкие температуры, ветра, полярные дни и ночи. Из-за этих условий у Антарктиды нет постоянного населения, что безусловно замедляет исследования. Однако, изучение этого континента показало, что он является очень ценным, и его изучение приносит плоды. 
 
 В настоящей главе проводится базовое ознакомление с Антарктидой, краткий исторический обзор изучения проблемы, затем рассказывается о современном состоянии исследований. 
 
 %Данная глава представляет собой введение в область исследований динамики ледяных покровов Антарктиды. Сначала дается краткий обзор изучаемых объектов и методов изучения, после следует краткий исторический экскурс изучения протекающих процессов. 
 
 \subsection{Антарктида}
 \subsubsection{Общие сведения}
Антарктида -- континент, расположенный на самом юге Земли. Центр Антарктиды примерно совпадает с Южным географическим полюсом. Омывается Атлантическим, Индийским и Тихим океанами. Некоторые ученые называют воды, омывающие Антарктиду Южным океаном.  \\

\begin{center}
   \includegraphics[scale = 0.25]{pic1.jpg}  \\
   \small \caption{Рис. 1: Спутниковое составное изображение Антарктиды}\\ \, \\
\end{center}


Площадь континента составляет около 14 107 000 $\textit{км}^2$, из них шельфовые ледники -- 930 000 $\textit{км}^2$. Средняя высота поверхности Антарктиды самая большая из всех континентов -- более 2000 м, а в центре континента достигает 4000 м. Большую часть этой высоты составляет постоянный ледниковый покров континента, под который скрыт континентальный рельеф, и лишь 0.3$\%$ (около 40 000 $\textit{км}^2$) её площади свободны ото льда. 


\subsubsection{Общий рельеф}
Антарктида -- самый высокий континент Земли, средняя высота поверхности континента над уровнем моря составляет более 2000 м., а в центре континента достигает 4000 м. Большую часть этой высоты составляет ледниковый покров, под которым скрыт континентальный рельеф. 

\begin{center}
    \includegraphics[scale = 0.4]{pic2.png} \\
    \small \caption{Рис. 2: Рельеф Антарктиды с учётом поднятия земной коры после таяния ледникового покрова и повышения уровня моря}
\end{center}


Через почти весь материк проходят Трансантарктические горы, которые делят Антарктиду на две части -- Западную Антарктиду и Восточную Антарктиду, имеющие различное происхождение и геологическое строение. На востоке находится высокое(наибольшее возвышение 4004 м. над уровнем моря) Советское плато. Западная часть состоит из группы гористых островов, соединённых между собой льдом. На тихоокеанском побережье расположены Антарктические Анды, высота которых превышает 4000 м; самая высокая точка континента -- 4892 м. над уровнем моря -- массив Винсон в горах Элсуорт. В Западной Антарктиде находится и глубочайшая депрессия континента -- впадина Бентли, вероятно, рифтового происхождения. Глубина впадины Бентли, заполненной льдом, достигает 2555 м. ниже уровня моря. 

\subsubsection{Подледный рельеф}

Исследования с помощью современных методов позволили больше узнать о подлёдном рельефе Антарктиды. В результате исследований выяснилось, что около трети материка лежит ниже уровня мирового океана, исследования также показали наличие горных цепей и массивов. \\

\begin{center}
    \includegraphics[scale = 0.2]{pic3.jpg}\\
    \small \caption{Рис. 3: Рельеф поверхности материка без ледникового покрова}
\end{center}

Новая карта НАСА, названная BedMashine Antarctica, сочетающая в себе сейсмические измерения и измерения движения льда, создала наиболее детальную карту материка, скрытого подо льдами Антарктики. Новая карта раскрывает неизвестные ранее топографические особенности, которые формируют ледяной поток. В данные включены свидетельства о самом глубоком каньоне на нашей планете. Изучая, сколько льда протекает через впадина Денмана за год, исследователи поняли, что он должен пролежать как минимум на 3500 м. ниже уровня моря, чтобы вместить весь объем замерзшей воды. 

Континент имеет сравнительно низкую вулканическую активность. Самый крупный вулкан -- гора Эребус на острове Росса в одноименном море. 

Исследования, проведенные НАСА, обнаружили в Антарктиде кратер астероидного происхождения. Диаметр воронки составляет 482 км. Кратер образовался при падении на Землю астероида поперечником примерно в 48 км. примерно 250 миллионов лет назад. Этот кратер считается крупнейшим на Земле и по одной из гипотез, пыль, поднятая при его образовании привела к многовековому похолоданию и гибели большей части флоры и фауны той эпохи. 

В случае полного таяния ледников, площадь Антарктиды сократится на треть: Западная Антарктида превратится в архипелаг, а Восточная Антарктида останется материком. 
\subsubsection{Ледниковый покров}
Когда-то Антарктида была обычным, свободным ото льда континентом, покрытым вечнозелёной растительностью. Резкое изменение климата, начавшееся 34 миллиона лет назад, запустило ряд процессов, приведших к падению количества CO$_2$ в атмосфере Земли, что вызвало похолодание и привело к почти полному оледенению Антарктиды. Также важную роль в оледенении Антарктиды сыграл разрыв перемычки, соединявшей Южную Америку и Антарктический полуостров, что привело к формированию антарктического циркумполярного течения и изоляции приантарктических вод и остальной части Мирового океана. 

Сейчас Антарктический ледяной щит является крупнейшим на нашей планете и превосходит ближайший по размеру гренландский ледниковый покров по площади приблизительно в 10 раз. В нём сосредоточено примерно 30 млн км$^3$ льда, то есть 90 \% всех льдов суши. Из-за тяжести льда, как показывают исследования геофизиков, континент просел, в среднем на 0.5 км, на что указывает и его относительно глубокий шельф. 

Ледниковый покров в Антарктиде содержит около 80 \% всех пресных вод планеты. 

Ледниковый щит имеет форму купола с увеличением крутизны поверхности к побережью, где он во многих местах обрамлён шельфовыми ледниками. Средняя толщина слоя льда -- 2500-2800 м, достигающая максимального значения в некоторых районах Восточной Антарктиды -- 4800 м. Накопление льда на ледниковом покрове приводит, как и в случае других ледников, к течению льда в зону абляции (разрушения), в качестве которой выступает побережье континента, где лёд откалывается в виде айсбергов. Годовой объем абляции оценивается в 2500 км$^3$.

При замерзании льда, в него вмерзают воздушные пузырьки, в результате лёд хранит состав атмосферного воздуха. Бурение льда в самой толстой части антарктического ледового покрова около станции Восток, начатое в 1950-х, дало возможность определить состояние климата Земли на протяжении последних 420000 лет.

Возраст ледникового покрова в верхней части можно определить по годовым слоям, состоящим из зимних и летних отложений, а также по маркирующим горизонтам, несущим информацию о глобальных событиях. Но на большой глубине для определения возраста используют численное моделирование растекания льда, которое строится, исходя из знания о рельефе, температуре, скорости накопления снега и тому подобного. 

 
\subsubsection{Ледяные потоки}
Ледяной поток -- участок быстро двигающегося льда, внутри ледяного покрова. Лед в леднике движется под давлением собственного веса. В глубину ледяной поток может достигать 2 километров. 
Для скорости потока очень важную роль играют осадки: чем мягче и более легко поддающиеся деформации осадки, тем выше может быть его скорость. У большинства потоков в основании находится слой воды, значительно снижающий силы трения, что также увеличивает скорость. Ещё одним фактором, влияющим на скорость является толщина потока: чем толще поток, тем больше движущее напряжение в слое, и, следовательно, скорость. 

Антарктический щит стекает в море благодаря нескольким ледяным потокам. Самый большой в восточной Антарктиде -- ледник Ламберта. В западной Антарктиде -- Пайн-Айланд и Туэйтс. Ледяные потоки серьезно влияют на повышение уровня моря, так как они отвечают за 90\% годовой потери массы.
 В то время как Восточная Антарктида весьма стабильна, потеря льда в Западной Антарктиде за последние 10 лет увеличилась на 59\%, а на Антарктическом полуострове на 140\%. 
 
 \\ \, \\
\begin{center}
   \includegraphics[scale=0.6]{pic4.jpg}\\
   \small \caption{Рис. 4: Скорости течения льда на поверхности Антарктического ледяного покрова. Данные получены с помощью спутниковой радиолокационной интерферометрии.}
\end{center}

 
Геоморфичекие особенности, такие как батиметрические впадины, указывают на то, где палео-ледяные потоки Антарктиды простирались во время последнего ледникового максимума. Анализ форм рельефа палео-ледяных потоков выявил значительную асинхронность в истории движения отдельных ледяных потоков. Осознание этого факта важно при рассмотрении того, как геоморфология, лежащая в основе ледяных потоков, определяет с какой скоростью и как они отступают. Кроме того, это повышает важность внутренних факторов, таких как характеристики дна, уклон и размер водосборного бассейна, в определении динамики ледяного потока. 

Антарктический лёд распространяется из нескольких центров к периферии покрова. В разных его частях это движение идёт с разной скоростью. В центре Антарктиды лёд движется медленно, у ледникового края его скорости возрастают до нескольких десятков, сотен, а в некоторых местах и тысяч метров в год. Быстрее всего двигаются ледяные потоки, которые погружаются в открытый океан. Как раз их скорости нередко достигают километра в год, а иногда и превосходят. Однако большинство ледяных потоков впадают не в океан, а в шельфовые ледники. Потоки такой категории движутся медленнее, их скорость не превышает 300 -- 800 метров в год. Такой медленный темп обычно объясняют сопротивлением со стороны шельфовых ледников, которые сами, как правило, тормозятся берегами и отмелями. Это предположение подтверждается наблюдениями, проведенными в 2002 году, после того, как от ледника Ларсена B откололся айсберг площадью свыше 3250 км$^2$ и толщиной 220 м. 4 ледника, питавшие ледяной щит значительно увеличили скорость своего движения. Это не могло быть связано с сезонной изменчивостью, т.к. ледники, впадающие в остатки шельфового ледника не укорялись.  






\subsection{История изучения ледяных потоков}
Исследования антарктических ледяных потоков начались в 1970-х. Полевые наблюдения и дистанционное зондирование позволили установить детальную конфигурацию и особенности поверхностного течения ледяных потоков Западной Антарктики: ледяного потока Уиллиса, ледяного потока Камба, ледника Туэйтса и других. Было высказано предположение, что до 90\% стока антарктического ледяного покрова происходит через быстро движущиеся ледяные потоки и выводные ледники. Несколько ледяных потоков были исследованы при помощи скважин, проплавленных системами бурения горячей водой для того, чтобы получить доступ к ложу ледника и обеспечить скважинно-геофизические наблюдения на границе ледника и подстилающих горных пород. Однако эти исследования единичны и не позволяют сделать однозначных выводов о механизме скольжения.

В настоящее время вопросу движения ледниковых покровов посвящено множество работ и проектов. Динамику льда изучают с помощью ГНСС, наблюдая за движением внешних слоев. Для более широкого изучения процессов, происходящих в Антарктиде, было запущено несколько спутниковых миссий: IceSat-1, IceSat-2, CryoSat. С их помощью ведутся наблюдения за балансом массы ледяного покрова, высоты облаков и для топографии.


\section{Описание используемых инструментов и технологий}
\subsection{Глобальная навигационная спутниковая система}

%ГНСС -- система, предназначенная для определения местоположения наземных, водных и воздушных объектов, а также низкоорбитальных космических аппаратов. Спутниковые системы навигации также позволяют получить скорость и направление движения приёмника сигнала. Кроме того, могут использоваться для получения точного времени. 

%На текущий момент времени действующими системами являются GPS, ГЛОНАСС, <<Бэйдоу>>, DORIS и Gallileo

%\subsubsection{Принцип работы}
%Принцип работы спутниковых систем навигации основан на измерении расстояния от антенны на объекте, координаты которого необходимо получить, до спутников, положение которых известно с большой точностью. Таблица положений всех спутников называется альманахом, которым должен располагать любой спутниковый приёмник до начала измерений. Обычно приёмник сохраняет альманах в памяти со времени последнего выключения, и если он не устарел -- мгновенно использует его. Каждый спутник передаёт в своём сигнале весь альманах. Таким образом, зная расстояния до нескольких спутников системы, с помощью обычных геометрических построений, на основе альманаха, можно вычислить положение объекта в пространстве. 

%Метод измерения расстояния от спутника до антенны приёмника основан на том, что скорость распространения радиоволн предполагается известной. Для осуществления возможности измерения времени распространяемого радиосигнала, каждый спутник навигационной системы излучает сигналы точного времени, используя точно синхронизированные с системным временем атомные часы. При работе спутникового приёмника, его часы синхронизируются с системным временем, и при дальнейшем приёме сигналов, вычисляется задержка между временем излучения, содержащимся в самом сигнале, и временем приёма сигнала. Располагая этой информацией, навигационный приёмник вычисляет координаты антенны. Все остальные параметры движения вычисляются на основе измерения времени, которое объект затратил на перемещение между двумя или более точками с определёнными координатами.

%\subsubsection{Применение систем навигации}
%Кроме навигации, координаты, получаемые благодаря спутниковым системам, используются в следующих отраслях:

%\begin{itemize}
%    \item Геодезия с помощью систем навигации определяются точные координаты точек
%    \item 
%\end{itemize}

ГНСС (Global Navigation Satellite System -- GNSS) -- спутниковые системы, используемые для определения местоположения в любой точке земной поверхности с применением специальных навигационных или геодезических приёмников. GNSS-технология нашла широкое применение в геодезии, городском и земельном кадастре, при инвентаризации земель, строительстве инженерных сооружений, в геологии, астрономии и т.д.

На текущий момент времени действующими системами являются GPS, ГЛОНАСС, <<Бэйдоу>>, DORIS и Gallileo

\subsubsection{Основные достоинства и преимущества}

\begin{itemize}
    \item Не требуется прямой видимости между пунктами 
    \item Благодаря автоматизации измерений, сведены к минимуму ошибки наблюдателей 
    \item Позволяет круглосуточно при любых погодных условиях определять координаты объектов в любой точке Земли 
    \item Точность GNSS-определений мало зависит от погодных условий 
    \item GNSS позволяет значительно сократить сроки проведения работ по сравнению с традиционными методами 
    \item GNSS-результаты предоставляются в цифровом виде и могут быть легко экспортированы в картографические или географические информационные системы (ГИС)
\end{itemize}

\subsubsection{Устройство и классификация приёмников}
Геодезический ГНСС-приёмник -- радиоприёмное устройство для определения географических координат текущего положения антенны приёмника, на основе данных о временных задержках прихода радиосигнала. 

Современный приёмник состоит из трёх основных элементов:
\begin{itemize}
    \item Приёмник -- основное устройство, которое получает информацию от спутников, обрабатывает её, а также производит запись в память или на внешнее устройство
    \item Антенна -- принимающий элемент
    \item Контроллер -- устройство, позволяющее управлять работой приёмника
\end{itemize}

По сложности технических решений и объему аппаратных затрат спутниковые приёмники разделяют на:
\begin{itemize}
    \item Одноканальный -- позволяет в текущий момент времени вести приём и обработку радиосигнала только одного спутника
    \item Многоканальный -- позволяет одновременно принимать и обрабатывать сигналы нескольких спутников
\end{itemize}

В настоящее время в основном выпускаются многоканальные приёмники. 

Кроме того, приёмники можно разделить на два типа:
\begin{itemize}
    \item Односистемный -- принимающий сигналы одной навигационной системы
    \item Многосистемный -- принимающий сигналы разных систем навигации
\end{itemize}

В зависимости от вида принимаемых и обрабатываемых сигналов, приёмники делятся на:
\begin{itemize}
    \item Одночастотный, кодовый
    \item Двухчастотный, кодовый
    \item Одночастотный кодово-фазовый
    \item Двухчастотный кодово-фазовый
\end{itemize}
Кодовые приёмники предназначены для определения трехмерного положения точки, скорости и направления движения. Они позволяют определять плановое положение точки, как правило, с точностью до единиц метров, а высотное положение определяется с точностью порядка 10 метров. Двухчастотные кодовые приёмники обеспечивают субметровую точность. А для повышения точности высотных измерений в приёмники встраивают баровысотомер. Эти приёмники удобны при выполнении полевых географических и геологических работ, так как на экране можно отобразить карту маршрута, определять своё местоположение, расстояние, направление и время прибытия к цели. Полученные результаты могут накапливаться и храниться в памяти прибора, а затем вводиться в компьютер для дальнейшей обработки. Эти приёмники имеют малые габариты и массу, работают в широком диапазоне температур и малоэнергоёмки. 

По точности спутниковые приёмники делятся на три класса:
\begin{itemize}
    \item Навигационный класс -- точность определения координат 150-200 м.
    \item Класс картографии и ГИС -- 1-5 м.
    \item Геодезический класс -- до 1 см. (1-3 см. в кинематическом режиме, до 1 см. при статических измерениях)
\end{itemize}

\subsubsection{Принцип работы}

Принцип работы спутниковых систем навигации основан на измерении расстояния от антенны на объекте (координаты которого нужно получить) до спутников, положение которых известно с большой точностью. Таблица положений всех спутников называется альманахом, которым должен располагать любой спутниковый приёмник до начала измерений. Обычно приёмник сохраняет альманах в памяти со времени последнего выключения и если он не устарел -- мгновенно использует его. Каждый спутник передаёт в своём сигнале весь альманах. Таким образом, зная расстояния до нескольких спутников системы, с помощью обычных геометрических построений, на основе альманаха, можно вычислить положение объекта в пространстве.

Метод измерения расстояния от спутника до антенны приёмника основан на том, что скорость распространения радиоволн предполагается известной. Для осуществления возможности измерения времени распространяемого радиосигнала каждый спутник навигационной системы излучает сигналы точного времени, используя точно синхронизированные с системным временем атомные часы. При работе спутникового приёмника, его часы синхронизируются с системным временем, и при дальнейшем приёме сигналов вычисляется задержка между временем излучения, содержащимся в самом сигнале, и временем приёма сигнала. Располагая этой информацией, навигационный приёмник вычисляет координаты антенны. Все остальные параметры движения (скорость, курс, пройденное расстояние) вычисляются на основе измерения времени, которое объект затратил на перемещение между двумя или более точками с определёнными координатами.

\subsubsection{Основные элементы}
Основные элементы спутниковой системы навигации:
\begin{itemize}
    \item спутниковая  группировка -- орбитальная группировка спутников, излучающих специальные радиосигналы
    \item наземный сегмент -- наземная система управления и контроля, включающая блоки измерения текущего положения спутников и передачи на них полученной информации для корректировки информации об орбитах
    \item спутниковые навигаторы -- аппаратура потребителя спутниковых навигационных систем, используемая для определения координат
    \item опционально: наземная система радиомаяков, позволяющая значительно повысить точность определения координат
    \item опционально: информационная радиосистема для передачи пользователям поправок, позволяющих значительно повысить точность определения координат (с середины 2010-х является неотъемлемой частью ГНСС)
\end{itemize}

\subsubsection{Действующие спутниковые системы}
\paragraph{GPS}
\\ \, \\
 Принадлежит министерству обороны США. Этот факт, по мнению некоторых государств, является её главным недостатком. Устройства, поддерживающие навигацию по GPS, являются самыми распространенными в мире. Также известная под более ранним названием NAVSTAR. Была запущена в феврале 1978 г. 
    
    Обеспечивает измерение расстояния, времени и определяющая местоположение во всемирной системе координат WGS 84. Спутниковая группировка этой системы обращается вокруг Земли по круговым орбитам с одной высотой и периодом обращения для всех спутников. Круговая орбита с высотой около 20200 км. (радиус орбиты около 26600 км.) является орбитой суточной кратности с периодом обращения с 11 часов 58 минут. Таким образом, спутник совершает два витка вокруг Земли за одни звёздные сутки. Наклонение орбиты (55$^o$) является также общим для всех спутников системы. Единственным отличием орбит спутников является долгота восходящего узла, она изменяется с шагом приблизительно в 60$^o$. Таким образом, несмотря на почти одинаковые параметры орбит, спутники обращаются в шести различных плоскостях, по 4 аппарата в каждой.  
    
    Спутники излучают открытые для использования сигналы в диапазонах: L$_1$ = 1575.42 МГц и L$_2$ = 1227.60 МГц, а последние модели излучают также на частоте L$_5$ = 1176.45 МГц. Эти частоты являются соответственно 154-й, 120-й и 115-й гармониками фундаментальной частоты 10.23 МГц, генерируемой бортовыми атомными часами спутника с суточной нестабильностью не хуже 10$^(-13)$; при этом частота атомных часов сдвинута, чтобы компенсировать релятивистский сдвиг, обусловленный движением спутника относительно наземного наблюдателя и разностью гравитационных потенциалов спутника и наблюдателя на поверхности Земли. С запуском спутников блока IIF введена новая частота L$_5$, этот сигнал также называют "Safety of life" (охрана жизни человека). Сигнал на этой частоте мощнее на 3 децибела, чем гражданский сигнал, имеет полосу пропускания в 10 раз шире. Сигнал используется в критических ситуациях, связанных с угрозой для жизни человека. 
    
    Сигналы модулируются псевдослучайными последовательностями двух типов: C/A-код и P-код. C/A (CLear Access) ---  общедоступный код с периодом повторения 1023 цикла и частотой следования импульсов 1.023 МГц. Именно с этим кодом работают все гражданские GPS-приёмники. C/A-код передается на частоте L$_1$.
    P-код (Protected/Precise) используется в закрытых для общего пользования системах, период его повторения составляет $2 \times 10^(14)$ циклов. Сигналы, модулированные P-кодом, передаются на двух частотах: L$_1$ и L$_2$. 
    
    24 спутника обеспечивают полную работоспособность системы в любой точке земного шара, но не всегда могут обеспечить уверенный приём и хороший расчёт позиции, поэтому, для увеличения точности позиционирования и резерва на случай сбоев, общие число спутников на орбите поддерживается в большем количестве (обычно 32).
    
    Слежение за орбитальными аппаратами осуществляется с помощью главной контрольной станции и 10 станций слежения. Главная станция расположена на авиабазе ВВС США Фалькон, штат Колорадо. Остальные станции слежения разбросаны по миру. Некоторые станции способны посылать на спутники корректировочные данные в виде радиосигналов с частотой 2000 -- 4000 МГц. Спутники последнего поколения способны распределять полученные данные среди других спутников.  
    
\paragraph{ГЛОНАСС} 

\\ \, \\ Принадлежит министерству обороны РФ. Разработка системы началась в 1976 г., полное развёртывание системы завершилось в 1995 г. После 1996 г. спутниковая группировка сокращалась и к 2002 г. пришла в упадок. Была восстановлена к концу 2011 г. 
    
    Спутники ГЛОНАСС находятся на средневысотной круговой орбите на высоте 19400 км. с наклонением 64.8$^o$ и периодом 11 часов 15 минут. Такая орбита оптимальная для использования в высоких широтах. Спутниковая группировка развёрнута в трёх орбитальных плоскостях, с 8 равномерно распределёнными спутниками в каждой. Для обеспечения глобального покрытия необходимы 24 спутника, в то время как для покрытия территории России необходимы 18 спутников. Сигналы передаются с направленностью 38$^o$ с использованием правой круговой поляризации, мощностью 318 -- 500 Вт). Для определения координат приёмник должен принимать сигнал как минимум четырёх спутников и вычислить расстояния до них. При использовании трёх спутников определение координат затруднено из-за ошибок, вызванных неточностью часов приёмника. 
    
    Используются два типа навигационных сигналов: открытые с обычной точностью и защищенные с повышенной точностью. Сигналы передаются методом расширения спектра в прямой последовательности (DSSS) и модуляцией через двоичную фазовую манипуляцию (BPSK). Все спутники используют одну и ту же псевдослучайную кодовую последовательность для передачи открытых сигналов, однако каждый спутник передаёт на разной частоте, используя 15-канальное разделение по частоте (FDMA). Сигнал в диапазоне L$_1$ находится на центральной частоте 1602 МГц, а частота передачи спутников определяется по формуле 1602 МГц $+ n \times $0.5625 МГц, где $n$ --- номер частотного канала ($n = -7, -6, ... , 5, 6$). Сигнал в диапазоне L$_2$ находится на центральной частоте 1246 МГц, а частота каждого канала определяется по формуле 1246 МГц $+ n \times $ 0.4375 МГц. Противоположно расположенные аппараты не могут быть одновременно видны с поверхности Земли, поэтому 15 радиоканалов достаточно для 24 спутников. 
    
    Открытый сигнал генерируется через сложение по модулю 2 трёх кодовых последовательностей: псевдослучайного дальномерного кода со скоростью 511 кбит/с, навигационного сообщения со скоростью 50 бит/с, и 100 Гц манчестер-кода. Все эти последовательности генерируются одним тактовым генератором. Псевдослучайный код генерируется 9-шаговым сдвиговым регистром с периодом 1 мс. 
    
    Наземный сегмент управления ГЛОНАСС почти полностью расположен на территории России и состоит из:
    \begin{itemize}
        \item двух центров управления системой
        \item пяти центров телеметрии, слежения и управления
        \item двух лазерных дальномерных станций
        \item десяти контрольно-измерительных станций
    \end{itemize}
    
    Ошибки навигационных определений ГЛОНАСС по долготе и широте составляют 3-6 м. в зависимости от точки приёма. В предельном случае эта система способна обеспечить определение местонахождения объекта с точностью до 2.8 метра. 
    
\paragraph{Бэйдоу}
 \\ \, \\
 Китайская глобальная спутниковая система навигации, основанная на геостационарных, геосинхронных спутниках и спутниках со средними орбитами. Реализация программы началась в 1994 году. По состоянию на 2015 год система имела 14 работающих спутников: 5 на геостационарных орбитах, 5 на геосинхронных и 4 на средних околоземных. Средневысотные спутники находятся на круговой орбите на высоте 21500 м. В 2020 году был запущен 55-ый спутник системы, тем самым было завершено создание глобальной спутниковой системы навигации. 
 
 Создававшаяся с 1994 года система <<Бэйдоу-1>> была завершена 21 декабря 2000 года, после запуска двух необходимых для неё спутников. Система основывалась на идее о достаточности двух спутников на геосинхронной орбите для определения местоположения на ограниченной территории, при этом в качестве третьего, неподвижного виртуального спутника рассматривался центр Земли. 
 
 Система <<Бэйдоу-2>> начала создаваться с 2004 года. Она была запущена в коммерческую эксплуатацию в 2012 году как региональная система позиционирования, при этом спутниковая группировка составляла 16 спутников, 14 из которых входило в данную систему.
 
 Система <<Бэйдоу-3>> начала создаваться в 2009 году. Глобальные базовые навигационные услуги она начала предоставлять в 2018 году, а в 2020 было полностью завершено её создание, а вместе с ней и всей глобальной навигационной системы <<Бэйдоу>>. 
 
 Система <<Бэйдоу>> предоставляет семь услуг:
 \begin{itemize}
     \item определение координат и сигналы атомных часов
     \item глобальная передача кратких сообщений (14 кбит или тысяча китайских иероглифов)
     \item региональная передача кратких сообщений
     \item подключение к поисково-спасательной системе
     \item спутниковая коррекция и контроль 
     \item коррекция с помощью наземных станций
     \item высокоточное позиционирование
 \end{itemize}
 
 Станции слежения оборудованы двухчастотными приёмниками и антеннами, которые способны принимать сигналы систем GPS. 7 из них размещены в Китае, и ещё 5 в Сингапуре, Австралии, ОАЭ, Европе и Африке. 
 
 По всему миру точность определения координат достигает 10 метров. Использование большой сети базовых станций на территории КНР (более трёх тысяч станций) позволило достичь точности нескольких сантиметров в реальном времени и нескольких миллиметров при режиме накопления информации. 
    
\paragraph{DORIS}
\\ \, \\
Французская навигационная система. Принцип работы связан с применением эффекта Допплера. В отличие от других спутниковых навигационных систем, основана на системе стационарных наземных передатчиков, а приёмники расположены на спутниках. После определения точного положения спутника, система может установить точные координаты и высоту маяка на поверхности Земли. Первоначально предназначалась для наблюдения за океанами и дрейфом материков. Количество спутников-носителей DORIS не ограничено. Результаты измерений, предоставляемые приёмниками DORIS, могут использоваться в следующих приложениях:
 \begin{itemize}
     \item Поддержка POD (Precise Orbit Determination --- точное определение орбит) для альтиметрии и других миссий
     \item контроль орбиты (на борту или на Земле)
     \item позиционирование наземного маяка
     \item геофизическое моделирование (земное гравитационное поле, атмосфера, ионосфера, мониторинг движения полюсов Земли и т.д.)
     \item контроль целостности системы DORIS 
 \end{itemize}
 
 В основе системы заложено точное измерение доплеровского сдвига радиочастоты сигналов, передаваемых наземными маяками и принимаемых на борту космического аппарата. Измерения производятся на двух частотах: 2.03625 ГГц --- для измерения доплеровского сдвига, и 401.25 МГц --- для коррекции задержки распространения сигнала в ионосфере. Вторая частота также используется для отметок времени измерений и передачи вспомогательных данных. 
 
 Замеры сдвига частоты производятся на борту спутника каждые 10 секунд. Полученная радиальная скорость (её точность примерно равна 0.4 мм/с) используется на Земле в комбинации с динамической моделью траектории спутника для точного определения орбиты с ошибкой по высоте не более 5 см. 

\paragraph{Galileo}
\\ \, \\
Совместный проект спутниковой системы навигации Европейского союза и Европейского космического агентства. Система предназначена для решения геодезических и навигационных задач. 

Спутники Galileo выводятся на круговые геоцентрические орбиты высотой 23222 км. (радиусом 29600 км.), проходят один виток за 14 ч. 4 мин. 42 с. и обращаются в трёх плоскостях, наклонённых под углом 56$^o$ к экватору. Долгота восходящего узла каждой из трёх орбит отстоит на 120$^o$ от двух других. На каждой из орбит находится 8 действующих и 2 резервных спутника. Эта конфигурация спутниковой группировки обеспечит одновременную видимость из любой точки земного шара по крайней мере четырех аппаратов. Временная погрешность атомных часов, установленных на спутниках, составляет одну миллиардную долю секунды, что обеспечивает точность определения места приёмника около 30 см. на низких широтах. За счёт более высокой, чем у спутников GPS, орбиты, на широте Полярного круга обеспечивается точность до одного метра. 
Система использует систему координат Galileo Terrestrial Reference Frame (GTRF), связанную с ITRF и определенную таким образом, что её расхождение с ITRF не превышает 3 см. с вероятностью 0.95.

\subsubsection{Применение систем навигации}
Координаты, получаемые благодаря спутниковым системам, используются в следующих отраслях:
\begin{itemize}
    \item Геодезия: определение точных координат точек
    \item Навигация: осуществление морской и дорожной навигации
    \item Спутниковый мониторинг транспорта: ведение наблюдения за положением, скоростью автомобилей, контроль за их движением
    \item Сотовая связь: В некоторых странах мобильные телефоны с GPS используются для оперативного определения местонахождения человека, звонившего 911. В России аналогичным проектом является Эра-ГЛОНАСС
    \item Тектоника: с помощью систем навигации ведутся наблюдения движений и колебаний плит
    \end{itemize}
    


\subsection{Спутниковая альтиметрия}
Спутниковый высотомер --- неотъемлемая часть современных комплексов дистанционного зондирования и мониторинга Земли из космоса. В основе работы подобных приборов лежит традиционный принцип импульсной радиолокации, состоящий в извлечении информации о расстоянии до подстилающей поверхности и запаздывания отраженного импульса относительно излучаемого. Спектр приложений спутниковой альтиметрии весьма широк и разнообразен: данные, полученные в результате измерений, чрезвычайно важны для решения геофизических, океанографических, экологических и ряда других задач. 


%Спутниковая альтиметрия измеряет расстояние между спутником и целью на Земле. Обычно это делается с помощью радиолокационной альтиметрической системы, которая посылает радиолокационный импульс на поверхность Земли, а затем измеряет время, что требуется импульсу, чтобы достичь поверхности и вернуться для оценки расстояния. Конкретные характеристики сигнала, такие как величина и форма сигнала, дают информацию о типе исследуемой поверхности. Для проведения измерений используются специальные приборы --- лидары.

%\subsubsection{Лидар}
%Лидар (LIDAR --- Light Detection and Ranging --- обнаружение и определение расстояния с помощью света) --- технология измерения расстояний путем излучения света (лазер) и замера времени возвращения этого отражённого света на приёмник. 

%\paragraph{Устройство лидара}
%\\ \, \\
%В большинстве конструкций излучателем служит лазер, формирующий короткие импульсы света высокой мгновенной мощности. Модулирующая частота выбирается так, чтобы пауза между двумя последовательными импульсами была не меньше, чем время отклика от цели. Выбор длины волны зависит от функции лазера и требований безопасности и скрытности прибора. Наиболее часто применяются Nd:YAG лазеры (твердотельный лазер с активной средой из алюмо-иттриевого граната (Y$_3$Al$_5$O$_(12)$), легированного ионами неодима (Nd)) и длины волн:
%\begin{itemize}
 %   \item 1550 нм --- инфракрасное излучение, невидимое ни глазу человека, ни распространенным приборам ночного видения. Глаз не способен сфокусировать этот свет на сетчатке, поэтому травматический порог этой длины волны существенно выше, чем для более коротких волн. Однако на деле, риск повреждения глаза существенно выше, т.к. глаз не реагирует на ИК излучение, и не защищает себя. 
%    \item 1064 нм --- ближнее инфракрасное излучение неодимовых и иттребиевых лазеров. Не видно глазу, но обнаруживается приборами ночного видения. 
%    \item 532 нм --- зелёное излучение неодимового лазера, эффективно распространяется в воде. 
%\end{itemize}












\subsection{Спутниковые миссии для изучения динамики льда}

\subsubsection{IceSat-1}

Для изучения динамики ледяного покрова используют разные методы, в том числе и спутниковые миссии.

Первой спутниковой миссией была ICESAT (Ice, Cloud, and land Elevation SATellite). Это спутник NASA для измерения баланса массы ледяного покрова, высоты облаков, а также топографии суши и растительности, он работал как часть системы наблюдения за Землёй (EOS). ICESAT был запущен 13 января 2003 года с ракеты-носителя Delta II с базы ВВС Ванденберг в Калифорнии на почти круговую околополярную орбиту, высотой примерно 600 км.  Он проработал семь лет, прежде чем был выведен из эксплуатации в феврале 2010 года, после того, как спутник перестал выполнять свои функции и ученые не смогли его восстановить. 

Миссия  ICESAT была разработана для получения данных о высоте, необходимых для определения баланса массы ледяного покрова, а также информации о свойствах облаков, особенно стратосферных, распространенных над полярными областями. Он предоставлял данных о топографии и растительности по всему миру, в дополнение к полярному покрытию ледяных щитов Гренландии и Антарктики. Спутник был признан полезным для оценки важных характеристик леса, включая густоту деревьев. 

Единственным прибором на ICESAT была геофизическая лазерная альтиметрическая система (Geoscience Laser Altimeter System -- GLAS). GLAS объединил в себе высокоточный наземный лидар с чувствительным двухволновым облачным и аэрозольным лидаром. Лазеры GLAS излучали инфракрасные и видимые импульсы с длинами волн 1064 и 532 нм. Пока ICESAT находился на орбите, GLAS произвела серию замеров пятнами диаметром примерно 70 метров, между которыми было почти 170 метров. На этапе ввода в эксплуатацию, спутник был выведен на орбиту, которая повторяла наземный трек каждые 8 дней. Для основной части миссии в течение августа и сентября 2004 года спутник был переведен на орбиту, с периодом повтора трека 91 день.

ICESAT был рассчитан на работу от трех до пяти лет. Испытания показали, что каждый лазер GLAS должен прослужить два года, то есть система должна иметь три лазера для выполнения номинальной продолжительности миссии. Во время первых испытаний на орбите 29 марта 2003 года преждевременно вышел из строя a pump diode module?  на первом лазере GLAS. Расследование показало, что коррозионная деградация of the pump diodes? произошла из-за неучтенной, но известной реакции между индиевым припоем и золотыми проводами снизила надежность лазеров. Из-за этого, общий срок службы GLAS должен был составить менее года. После двух месяцев полной эксплуатации осенью 2003 года оперативный план наблюдений был изменен, инструмент использовался в течение месячных периодов каждые три-шесть месяцев, чтобы расширить временные ряды измерений, особенно для ледяных щитов. Последний лазер вышел из строя 11 октября 2009 года, и после попыток его перезапуска в феврале 2010 года, спутник был выведен из эксплуатации. В период с 23 июня по 14 июля  2010 года, спутник был переведен на более низкую орбиту, для ускорения затухания орбиты. 14 августа 2010 года он был выведен из эксплуатации и в 08:49 UTC 30 августа 2010 года спутник вошел в атмосферу.

\subsubsection{IceSat-2}

 Продолжателем миссии ICESAT стал спутник ICESAT-2. Этот спутник представляет собой систему для измерения высоты ледяного покрова и толщины морского льда, а также топографии суши, характеристик растительности и облаков. Входит в EOS. ICESAT-2 был запущен 15 сентября 2018 года с базы ВВС Ванденберг в Калифорнии ракетой Delta II на почти круговую околополярную орбиту с высотой примерно 496 км. Он был рассчитан на работу в течение трех лет, с запасом топлива на семь лет. Спутник вращается вокруг Земли со скоростью 6.9 километра в секунду. 
 
Миссия ICESAT-2 предназначена для получения данных о высоте, необходимых для определения баланса массы ледяного покрова, а также информации о растительном покрове. Он обеспечивает измерения топографии городов, озер и водохранилищ, океанов и поверхности суши по всему земному шару, в дополнение к полярному покрытию. Спутник также может определять топографию морского дна на глубине до 30 метров в чистых водах прибрежных районов. 
Единственным прибором на ICESAT-2 является усовершенствованная лазерная топографическая альтиметрическая система (ATLAS).  ATLAS излучает видимые лазерные импульсы длиной волны 532 нм. В процессе работы, система генерирует шесть лучей, связанные в три пары, чтобы лучше определять наклон поверхности и обеспечивать большее покрытие земли. Each beam pair is 3.3 km apart across the beam track, and each beam in a pair is separated by 2.5 km along the beam track. The laser array is rotated 2 degrees from the satellite's ground track so that a beam pair track is separated by about 90 m? . Частота лазерных импульсов в сочетании со скоростью спутника приводит к тому, что ATLAS замеряет высоту каждые 70 см вдоль наземного трека. 
Лазер срабатывает с частотой 10 кГц. Обратный импульс улавливается с помощью бериллиевого телескопа диаметром 79 см. Бериллий обладает высокой удельной прочностью и сохраняет форму в широком диапазоне температур. Телескоп собирает фотоны с длиной волны 532 нм, таким образом отфильтровывая посторонний свет в атмосфере. Компьютерные программы дополнительно фильтруют входящий поток, оставляя для анализа только отраженные фотоны лидара. 
Примечательным атрибутом ATLAS является то, что инженеры позволили спутнику контролировать его положение в космосе. Они сконструировали лазерную систему отсчета, которая подтверждает настройку лазера в соответствии с телескопом. 

Управляет научными данными ICESAT-2 Национальный центр данных по снегу и льду (The National Snow and Ice Data Center).

ICESAT-2 преследует четыре научные цели:
\begin{itemize}
    \item Количественно оценить вклад полярного ледяного покрова в текущее и недавнее изменение уровня моря и связь с климатическими условиями;
    \item Количественная оценка региональных признаков изменений ледникового покрова для оценки механизмов, вызывающих эти изменения и улучшения прогнозных моделей ледникового покрова. Это включает в себя количественную оценку региональной эволюции изменений ледникового покрова;

    \item Оценить толщу морского льда для изучения обмена энергией, массой и влажностью между льдом, океаном и атмосферой;

    \item Измерение высоты растительного покрова, как основу для оценки крупномасштабных изменений биомассы.;
\end{itemize}
Кроме того, ICESAT-2 выполняет измерения облаков и аэрозолей, высоты океанов, внутренних водоемов (водохранилищ и озер), городов/, а также движения земли после землетрясений или оползней. 

Запуск ICESAT-2 состоялся 15 сентября 2018 года в 15:02 UTC. Чтобы поддержать непрерывность данных между выводом из эксплуатации ICESAT и запуском ICESAT-2, воздушная операция NASA IceBridge использовала различные самолеты для сбора данных о полярной топографии и измерения толщины льда с помощью комплектных лазерных высотомеров, радаров и других систем. 

ICESAT-2 всё ещё находится на орбите и передает данные. 


\subsubsection{CryoSat-1 и CryoSat-2}
Другим проектом по изучению полярного льда является CryoSat. Это программа Европейского Космического Агентства (ESA) для мониторинга изменений протяженности и толщины полярного льда с помощью спутника на низкой околоземной орбите.  Предоставленная информация о поведении прибрежных льдов, будет иметь ключевое значение для более точных прогнозов изменения уровня моря. Аппарат CryoSat-1 был потерян в результате неудачного запуска в 2005 году, однако программа была возобновлена успешным запуском замены, CryoSat-2, запущенным 8 апреля 2010 года. 

CryoSat управляется Европейским центром космических операций (European Space Operations Centre -- ESOC) в Дармштадте, Германия.
Основным прибором CryoSat является SIRAL (SAR (synthetic-aperture radar?)  Interferometric radar altimeters/ интерферометрический радарный альтиметр). SIRAL работает в одном из трех режимов, в зависимости от того, где (над поверхностью Земли) летел CryoSat. Над океанами и ледяным покровом спутник работает как традиционный радарный высотомер. Над морским льдом когерентно передаваемые эхосигналы комбинируются (обработка синтетической апертуры) для уменьшения воздействия на поверхность, чтобы CryoSat мог отображаться более мелкие льдины. Самый продвинутый режим CryoSat используется вокруг границ ледяного покрова и над горными ледниками. Здесь альтиметр выполняет обработку синтетической апертуры и использует вторую антенну в качестве интерферометра для определения поперечного угла до самого раннего отраженного сигнала радара. Это обеспечивает точное местоположение измеряемой поверхности, когда поверхность наклонена. 

Хотя CryoSat-2 в значительной степени такой же, как и исходный спутник, в него был включен ряд ключевых улучшений. Самым значительным было решение поставить полностью дублированную полезную нагрузку, чтобы позволить продолжить миссию, если неисправность приведет к потере радара SIRAL. Другие изменения были вызваны устареванием первоначальной конструкции, некоторые из них повысили надежность, а другие упростили работу спутника. Несмотря на все изменения, миссия остается прежней, а характеристики с точки зрения измерительных возможностей и точности остаются прежними. CryoSat-2 был запущен 8 апреля 2010 года в 13:57 UTC с космодрома Байконур на ракете "Днепр".

Второй инструмент, DORIS (Doppler Orbit and Radio Positioning Integration by Satellite) используется для точного расчета орбиты космического корабля. Множество ретрорефлекторов также установлено на борту, что позволяет проводить измерения с земли для проверки орбитальных данных, предоставленных DORIS.

После запуска CryoSat-2 был выведен на низкую околоземную орбиту с перигеем 720 км, апогеем 732 км, наклоном 92 градуса и периодом обращения 99.2 минуты. При запуске он имел массу 750 кг, и, как ожидается, проработает не менее трех лет. 


\section{Полученные данные}
\subsection{Источники данных}
Для изучения вопроса, поднятого в этой работе использовалось два источника данных: результаты замеров с помощью ГНСС-станции и данные спутниковых миссий. 
\subsubsection{Данные с ГНСС-пункта}
Для получения информации и местоположении пункта использовался навигационный приёмник Javad Triumph-1 с картой памяти. 
Для стабильной установки и бесперебойной работы приёмника, были проведены подготовительные работы. В итоге получили стационарную систему с радиопрозрачными защитными оболочками.  
\\ \, \\ 
\begin{center}
    \includegraphics[scale = 0.6]{pic5.jpg} \\
    \small \caption{Рис. 5: Приёмника на станции Восток с защитными оболочками.}
\end{center}

\subsubsection{Данные спутниковых миссий}
Углубляясь в изучение вопроса, были изучены данные спутниковых миссий IceSat-1, IceSat-2, CryoSat-2. Результаты, полученные спутником IceSat-1 не являются подходящими для этой работы, т.к. спутник завершил свою работу в 2010 году. Напротив, данные остальных миссий могут помочь посмотреть на полученные ГНСС-пунктом данные с другой стороны. 

\subsection{Данные с ГНСС-пункта на станции Восток}

Полученные значения были получены в формате JPS. Для обработки использовалось два метода: RTKLIB --- пакет программ с открытым исходным кодом для ГНСС-позиционирования и CSRS-PPP --- онлайн-сервис для постобработки данных ГНСС, позволяющее пользователями отправлять данные наблюдений через интернет. Сервис предоставляется правительством Канады на безвозмездной основе. 

\subsubsection{Обработка с помощью пакета RTKLIB}

RTKLIB работает только с данными в RINEX-формате, поэтому для начала данные нужно было перевести из формата JPS в подходящий. Для этого использовалось программное обеспечение JPS2RIN, конвертирующие файлы собственного формата JPS в формат RINEX. 
После проведенных преобразований, массив данных был полностью обработан, были получены данные широты, долготы и высоты пункта. Также с помощью метода наименьших квадратов был получен тренд каждой координаты. По получившимся результатам построены графики:

\begin{center}
    \includegraphics[scale=0.3]{pic6.jpg}\\
    \small \caption{Рис.6: Широта пункта. По оси абсцисс время в сутках, по оси ординат отложено отклонение пункта от изначального местоположения в метрах. }
    
    \includegraphics[scale=0.3]{pic7.jpg} \\
    \small \caption{Рис.7: Долгота пункта. По оси абсцисс время в сутках, по оси ординат отложено отклонение пункта от изначального местоположения в метрах. }
    
    \includegraphics[scale=0.3]{pic8.jpg} \\
    \small \caption{Рис.8: Высота пункта. По оси абсцисс время в сутках, по оси ординат отложена высота от уровня моря в метрах. }
\end{center}

Вычислив параметры тренда, была получена скорость движения пункта в каждом направлении:
\begin{itemize}
    \item Скорость движения по широте: $v_{lat} = -0.00378733587$ метров в сутки, то есть примерно 3.79 миллиметров в сутки
    \item Скорость движения по долготе: $v_{lon} = 0.00471147324$ метров в сутки, то есть примерно 4.71 миллиметров в сутки
    \item Скорость движения по высоте: $v_{height} = -0.00033501098$ метров в сутки, то есть примерно 0.34 миллиметра в сутки
\end{itemize}
По полученным скоростям можно понять, что станция движется на юго-восток и с каждым днём опускается всё ниже. 

Но кроме постоянного движения в указанном направлении, существует более мелкое движение, которое стало более заметным, когда был вычтен тренд:

\begin{center}
    \includegraphics[scale=0.85]{pic9.jpg}\\
    \small \caption{Рис.9: Широта пункта. По оси абсцисс время в сутках, по оси ординат отложена широта в метрах. }
    
    \includegraphics[scale=0.85]{pic10.jpg} \\
    \small \caption{Рис.10: Долгота пункта. По оси абсцисс время в сутках, по оси ординат отложена долгота в метрах. }
    
    \includegraphics[scale=0.85]{pic11.jpg} \\
    \small \caption{Рис.11: Высота пункта. По оси абсцисс время в сутках, по оси ординат отложена высота в метрах. }
\end{center}

Для более удобного рассмотрения, график был сглажен фильтром Гаусса. Видно, что имеет место некоторая периодическая зависимость. 

\subsubsection{Обработка с помощью сервиса CSRS-PPP}
Данный сервис так же работает только с данными в формате RINEX, но т.к. файлы уже были переведены в данный формат, дополнительных действий не потребовалось. Полученные широта, долгота и высота так же были обработаны методом наименьших квадратов для выявления тренда. 

\begin{center}
    \includegraphics[scale=0.85]{pic12.jpg}\\
    \small \caption{Рис.12: Широта пункта. По оси абсцисс время в сутках, по оси ординат отложено отклонение пункта от изначального местоположения в метрах. }
    
    \includegraphics[scale=0.85]{pic13.jpg} \\
    \small \caption{Рис.13: Долгота пункта. По оси абсцисс время в сутках, по оси ординат отложено отклонение пункта от изначального местоположения в метрах. }
    
    \includegraphics[scale=0.85]{pic14.jpg} \\
    \small \caption{Рис.14: Высота пункта. По оси абсцисс время в сутках, по оси ординат отложена высота от уровня моря в метрах. }
\end{center}

В данном рассмотрении также была получена скорость движения пункта в каждом направлении:

\begin{itemize}
    \item Скорость движения по широте: $v_{lat} = -0.00378942944$ метров в сутки, то есть примерно 3.79 миллиметров в сутки
    \item Скорость движения по долготе: $v_{lon} = 0.00470687628$ метров в сутки, то есть примерно 4.71 миллиметров в сутки
    \item Скорость движения по высоте: $v_{height} = -0.00033688586$ метров в сутки, то есть примерно 0.34 миллиметра в сутки
\end{itemize}

По этим результатам, станция так же движется на юго-восток и опускается.  

Вычтем тренд и посмотрим, проявится ли периодическая зависимость в этом варианте:

\begin{center}
    \includegraphics[scale=0.85]{pic15.jpg}\\
    \small \caption{Рис.15: Широта пункта. По оси абсцисс время в сутках, по оси ординат отложена широта в метрах. }
    
    \includegraphics[scale=0.85]{pic16.jpg} \\
    \small \caption{Рис.16: Долгота пункта. По оси абсцисс время в сутках, по оси ординат отложена долгота в метрах. }
    
    \includegraphics[scale=0.85]{pic17.jpg} \\
    \small \caption{Рис.17: Высота пункта. По оси абсцисс время в сутках, по оси ординат отложена высота в метрах. }
\end{center}

Видно, что некоторая периодичность здесь тоже прослеживается.

\subsubsection{Сравнение результатов, полученных с помощью RTKLIB и CSRS-PPP}
Сравнивая полученные значения скорости движения, мы видим, что их соответственное отклонение составляет меньше 1\%, то есть можно утверждать, что полученные скорости совпадают. 

Для наглядности сравнения методов обработки, были построены графики, показывающие разность значений, полученных разными методами:

\begin{center}
    \includegraphics[scale=0.7]{pic18.jpg}\\
    \small \caption{Рис.18: Разность методов по широте. По оси абсцисс время в сутках, по оси ординат отложена широта в метрах. }
    
    \includegraphics[scale=0.7]{pic19.jpg} \\
    \small \caption{Рис.19: Разность методов по долготе. По оси абсцисс время в сутках, по оси ординат отложена долгота в метрах. }
    
    \includegraphics[scale=0.7]{pic20.jpg} \\
    \small \caption{Рис.20: Разность методов по высоте. По оси абсцисс время в сутках, по оси ординат отложена высота в метрах. }
\end{center}


\subsection{Гипотеза о происходящих процессах}
В попытках объяснить наблюдаемые процессы, стоит рассматривать периодическое изменение координат и трендовое отдельно. 

Тренд может быть объяснен ледяным потоком, находящимся под станцией, который сдвигает и объекты, находящиеся на поверхности. 

Для выдвижения гипотезы о периодическом движении стоит заметить, что точки минимума графиков приходятся на середину года, то есть зимний период в южном полушарии. При падении температуры, лёд может становиться более плотным, что вызывает некоторый сдвиг. При наступлении летнего периода, это изменение происходит в обратную сторону: лёд расширяется, возвращая всё на исходные места. 


\section{Заключение}
В ходе данной работы было выполнено следующее:
\begin{itemize}
    \item Была проведена статистическая обработка координат ГНСС-пункта на станции Восток, в ходе которой была выявлена их периодичная зависимость от времени
    \item Были определены параметры движения ледника в районе станции Восток
    \item Была выдвинута гипотеза о механизме наблюдаемого движения ледникового покрова
\end{itemize}

\newpage

\section*{Список литературы}
\begin{enumerate}
    %\item  "Antarctica". The World Factbook. Central Intelligence Agency. 2011.
    %\item Дэвид Макгонигал, Лин Вудворт. Антарктика. Голубой континент = Antarctica. The Blue Continent / Романов А. П., Лебедев С. Л. — М.: Бертельсманн Медиа Москау АО, 2004. — С. 159. — 224 с.
    \item Anderson, John B. (2010). Antarctic Marine Geology. Cambridge: Cambridge University Press.
    \item Campbell, I.B.; Claridge, G.G.C., eds. (1987). "2: The Geology and Geomorphology of Antarctica". Antarctica: Soils, Weathering Processes and Environment. Developments in Soil Science. Vol. 16. Amsterdam: Elsevier.
    \item Day, David (2019). Antarctica: What Everyone Needs to know. Oxford: Oxford University Press.
    \item Joyner, Christopher C. (1992). Antarctica and the Law of the Sea. Dordrecht: Martinus Nijhoff Publishers.
    \item Pyne, Stephen J. (2017). The Ice: A Journey to Antarctica. University of Washington Press.
    \item "Break up of the Larsen Ice Shelf, Antarctica", NASA Earth Observatory. earthobservatory.nasa.gov
    \item Bell, Robin E.; Seroussi, Helene (2020). "History, mass loss, structure, and dynamic behavior of the Antarctic Ice Sheet". Science. 
    \item  "Quick Facts on Ice Shelves | National Snow and Ice Data Center". nsidc.org.
    \item  "China's GPS rival Beidou is now fully operational after final satellite launched". cnn.com. 24 June 2020.
    \item "GNSS signal - Navipedia". gssc.esa.int.
    \item Irene Klotz, Tony Osborne and Bradley Perrett (Sep 12, 2018). "The Rise Of New Navigation Satellites". Aviation Week & Space Technology.
    \item Kazmierski, Kamil; Zajdel, Radoslaw; Sośnica, Krzysztof (2020). "Evolution of orbit and clock quality for real-time multi-GNSS solutions". GPS Solutions.
    \item Seeber, Gunter (2003). Satellite geodesy. Berlin New York: Walter de Gruyter.
    \item "Decay Data: ICESAT". Space-Track. 30 August 2010.
    \item Schutz, B. E.; Zwally, H. J.; Shuman, C. A.; Hancock, D.; DiMarzio, J. P. (2005). "Overview of the ICESAT Mission". Geophysical Research Letters. NASA.
    \item "CryoSat - Earth Online". earth.esa.int. 
    \item Wade, Mark. "Cryosat". Encyclopedia Astronautica.
    \item  "CryoSat-2 Earth Explorer Opportunity Mission-2". www.esa.int
\end{enumerate}


\end{document}
