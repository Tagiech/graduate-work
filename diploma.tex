\documentclass[a4paper]{article}
\usepackage[12pt]{extsizes}
\usepackage[left=2cm,right=2cm,top=2.5cm,bottom=2.5cm,bindingoffset=0cm]{geometry}
\usepackage[utf8]{inputenc}
\usepackage[russian]{babel}
\usepackage{amsmath,amssymb}
\usepackage{mathtext}
\usepackage[T2A]{fontenc}
\linespread{1.2}
\usepackage{tikz}
\usepackage{pgfplots}
\usepackage{graphicx}
\graphicspath{{pictures/}} 


\begin{document}

\begin{titlepage}
\begin{center}
    {\bf
    Санкт-Петербургский государственный университет}
    
    \vspace{3cm}
    
    {\bf ТАГИЕВ Артур Мехман оглы 
    
    \, \\ 
    
    Выпускная квалификационная работа 
    
    \textit{Исследование динамики ГНСС-пункта \\ на антарктической станции Восток}}
    
    \vspace{2cm}
    Уровень образования: специалитет
    
    Направление 03.05.01 <<Астрономия>>
    
    Основная образовательная программа СМ.5012.2016 <<Астрономия>>
    
    Профиль <<Астрономия>>
\end{center}

\vspace{2cm}
\begin{flushright}
    \begin{tabular}{ll}
         & Научный руководитель: \\
         & доцент кафедры астрономии, \\
         & кандидат физико-математических наук \\
         & Петров Сергей Дмитриевич
         & \\
         & Рецензент:\\
         & научный сотрудник ГАО РАН \\
         & Щербакова Наталия Васильевна
    \end{tabular}
\end{flushright}

     \vfill
\begin{center}
    Санкт-Петербург 2022
\end{center} 
\end{titlepage}

\tableofcontents
\newpage

 \section{Введение}
 
 Данная работа посвящена континенту, открытие которого произошло последним -- Антарктиде. Открыли её 28 января 1820 года, и до сих пор Антарктида является самым неизученным континентом. Изучение этого материка осложнятся суровыми условиями: низкие температуры, ветра, полярные дни и ночи. Из-за этих условий у Антарктиды нет постоянного населения. Однако, изучение этого континента показало, что он является очень ценным, и его изучение приносит плоды. Так в Антарктиде 
 
 %Данная глава представляет собой введение в область исследований динамики ледяных покровов Антарктиды. Сначала дается краткий обзор изучаемых объектов и методов изучения, после следует краткий исторический экскурс изучения протекающих процессов. 
 
 \subsection{Антарктида}
 \subsubsection{Общие сведения}
Антарктида -- континент, расположенный на самом юге Земли. Центр Антарктиды примерно совпадает с Южным географическим полюсом. Омывается Антлантическим, Индийским и Тихим океанами. \\

\begin{center}
   \includegraphics[scale = 0.25]{pic1.jpg}  \\
   \small \caption{Рис. 1: спутниковое составное изображение Антарктиды}\\ \, \\
\end{center}


Площадь континента составляет около 14 107 000 $\textit{км}^2$, из них шельфовые ледники -- 930 000 $\textit{км}^2$. Средняя высота поверхности Антарктиды самая большая из всех континентов -- более 2000 м, а в центре континента достигает 4000 м. Большую часть этой высоты составляет постоянный ледниковый покров континента, под который скрыт континентальный рельеф, и лишь 0.3$\%$ (около 40 000 $\textit{км}^2$) её площади свободны ото льда. 

\subsubsection{Ледниковый покров}
Когда-то Антарктида была обычным, свободным ото льда континентом, покрытым вечнозелёной растительностью. Резкое изменение климата, начавшееся 34 миллиона лет назад, запустило ряд процессов, приведших к падению количества $CO_2$ в атмосфере Земли, что вызвало похолодание и привело к почти полному оледенению Антарктиды. Также важную роль в оледенении Антарктиды сыграл разрыв перемычки, соединявшей Южную Америку и Антарктический полуостров, что привело к формированию антарктического циркумполярного течения и изоляции приантарктических вод и остальной части Мирового океана. 

Сейчас Антарктический ледяной щит является крупнейшим на нашей планете и превосходит ближайший по размеру гренландский ледниковый покров по площади приблизительно в 10 раз. В нём сосредоточено примерно 30 млн $\textit{км}^3$ льда, то есть 90 \% всех льдов суши. Из-за тяжести льда, как показывают исследования геофизиков, континент просел, в среднем на 0.5 км, на что указывает и его относительно глубокий шельф. 

Ледниковый покров в Антарктиде содержит около 80 \% всех пресных вод планеты. 

Ледниковый щит имеет форму купола с увеличением крутизны поверхности к побережью, где он во многих местах обрамлён шельфовыми ледниками. Средняя толщина слоя льда -- 2500-2800 м, достигающая максимального значения в некоторых районах Восточной Антарктиды -- 4800 м. Накопление льда на ледниковом покрове приводит, как и в случае других ледников, к течению льда в зону абляции (разрушения), в качестве которой выступает побережье континента, где лёд откалывается в виде айсбергов. Годовой объем абляции оценивается в 2500 $\textit{км}^3$.

Возраст ледникового покрова в верхней части можно определить по годовым слоям, состоящим из зимних и летних отложений, а также по маркирующим горизонтам, несущим информацию о глобальных событиях. Но на большой глубине для определения возраста используют численное моделирование растекания льда, которое строится, исходя из знания о рельефе, температуре, скорости накопления снега и тому подобного. 



\subsection{Глобальная навигационная спутниковая система}

ГНСС -- система, предназначенная для определения местоположения наземных, водных и воздушных объектов, а также низкоорбитальных космических аппаратов. Спутниковые системы навигации также позволяют получить скорость и направление движения приёмника сигнала. Кроме того, могут использоваться для получения точного времени. 

На текущий момент времени действующими системами являются GPS, ГЛОНАСС, <<Бэйдоу>>, DORIS и Gallileo

\subsubsection{Принцип работы}
Принцип работы спутниковых систем навигации основан на измерении расстояния от антенны на объекте, координаты которого необходимо получить, до спутников, положение которых известно с большой точностью. Таблица положений всех спутников называется альманахом, которым должен располагать любой спутниковый приёмник до начала измерений. Обычно приёмник сохраняет альманах в памяти со времени последнего выключения, и если он не устарел -- мгновенно использует его. Каждый спутник передаёт в своём сигнале весь альманах. Таким образом, зная расстояния до нескольких спутников системы, с помощью обычных геометрических построений, на основе альманаха, можно вычислить положение объекта в пространстве. 

Метод измерения расстояния от спутника до антенны приёмника основан на том, что скорость распространения радиоволн предполагается известной. Для осуществления возможности измерения времени распространяемого радиосигнала, каждый спутник навигационной системы излучает сигналы точного времени, используя точно синхронизированные с системным временем атомные часы. При работе спутникового приёмника, его часы синхронизируются с системным временем, и при дальнейшем приёме сигналов, вычисляется задержка между временем излучения, содержащимся в самом сигнале, и временем приёма сигнала. Располагая этой информацией, навигационный приёмник вычисляет координаты антенны. Все остальные параметры движения вычисляются на основе измерения времени, которое объект затратил на перемещение между двумя или более точками с определёнными координатами.

\subsubsection{Применение систем навигации}
Кроме навигации, координаты, получаемые благодаря спутниковым системам, используются в следующих отраслях:

\begin{itemize}
    \item Геодезия с помощью систем навигации определяются точные координаты точек
    \item 
\end{itemize}











 
\subsection{Ледяные потоки}
Ледяной поток -- участок быстро двигающегося льда, внутри ледяного покрова. Лед в леднике движется под давлением собственного веса. За год перемещение может достигать 1000 метров в высоту, 50 километров в ширину и сотни километров в длину. В глубину ледяной поток может достигать 2 километров. В Антарктиде ледяные потоки отвечают за 90\% годовой потери массы.

Для скорости потока очень важную роль играют осадки: чем мягче и более легко поддающиеся деформации осадки, тем выше может быть его скорость. У большинства потоков в основании находится слой воды, значительно снижающий силы трения, что также увеличивает скорость. Ещё одним фактором, влияющим на скорость является толщина потока: чем толще поток, тем толще движущее напряжение в слое, и, следовательно, скорость. 

Антарктический щит стекает в море благодаря нескольким ледяным потокам. Самый большой в восточной Антарктиде -- ледник Ламберта. В западной Антарктиде -- Пайн-Айланд и Туэйтс. Ледяные потоки серьезно влияют на повышение уровня моря, так как они отвечают за 90\% годовой потери массы.
 В то время как Восточная Антарктида весьма стабильна, потеря льда в Западной Антарктиде за последние 10 лет увеличилась на 59\%, а на Антарктическом полуострове на 140\%. 
 
Геоморфичекие особенности, такие как батиметрические впадины, указывают на то, где палео-ледяные потоки Антарктиды простирались во время последнего ледникового максимума. Анализ форм рельефа палео-ледяных потоков выявил значительную асинхронность в истории движения отдельных ледяных потоков. Осознание этого факта важно при рассмотрении того, как геоморфология, лежащая в основе ледяных потоков, определяет с какой скоростью и как они отступают. Кроме того, это повышает важность внутренних факторов, таких как характеристики дна, уклон и размер водосборного бассейна, в определении динамики ледяного потока. 

\subsection{Спутниковые миссии для изучения динамики льда}

\subsubsection{IceSat-1}

Для изучения динамики ледяного покрова используют разные методы, в том числе и спутниковые миссии.

Первой спутниковой миссией была ICESAT (Ice, Cloud, and land Elevation SATellite). Это спутник NASA для измерения баланса массы ледяного покрова, высоты облаков, а также топографии суши и растительности, он работал как часть системы наблюдения за Землёй (EOS). ICESAT был запущен 13 января 2003 года с ракеты-носителя Delta II с базы ВВС Ванденберг в Калифорнии на почти круговую околополярную орбиту, высотой примерно 600 км.  Он проработал семь лет, прежде чем был выведен из эксплуатации в феврале 2010 года, после того, как спутник перестал выполнять свои функции и ученые не смогли его восстановить. 

Миссия  ICESAT была разработана для получения данных о высоте, необходимых для определения баланса массы ледяного покрова, а также информации о свойствах облаков, особенно стратосферных, распространенных над полярными областями. Он предоставлял данных о топографии и растительности по всему миру, в дополнение к полярному покрытию ледяных щитов Гренландии и Антарктики. Спутник был признан полезным для оценки важных характеристик леса, включая густоту деревьев. 

Единственным прибором на ICESAT была геофизическая лазерная альтиметрическая система (Geoscience Laser Altimeter System -- GLAS). GLAS объединил в себе высокоточный наземный лидар с чувствительным двухволновым облачным и аэрозольным лидаром. Лазеры GLAS излучали инфракрасные и видимые импульсы с длинами волн 1064 и 532 нм. Пока ICESAT находился на орбите, GLAS произвела серию замеров пятнами диаметром примерно 70 метров, между которыми было почти 170 метров. На этапе ввода в эксплуатацию, спутник был выведен на орбиту, которая повторяла наземный трек каждые 8 дней. Для основной части миссии в течение августа и сентября 2004 года спутник был переведен на орбиту, с периодом повтора трека 91 день.

ICESAT был рассчитан на работу от трех до пяти лет. Испытания показали, что каждый лазер GLAS должен прослужить два года, то есть система должна иметь три лазера для выполнения номинальной продолжительности миссии. Во время первых испытаний на орбите 29 марта 2003 года преждевременно вышел из строя a pump diode module?  на первом лазере GLAS. Расследование показало, что коррозионная деградация of the pump diodes? произошла из-за неучтенной, но известной реакции между индиевым припоем и золотыми проводами снизила надежность лазеров. Из-за этого, общий срок службы GLAS должен был составить менее года. После двух месяцев полной эксплуатации осенью 2003 года оперативный план наблюдений был изменен, инструмент использовался в течение месячных периодов каждые три-шесть месяцев, чтобы расширить временные ряды измерений, особенно для ледяных щитов. Последний лазер вышел из строя 11 октября 2009 года, и после попыток его перезапуска в феврале 2010 года, спутник был выведен из эксплуатации. В период с 23 июня по 14 июля  2010 года, спутник был переведен на более низкую орбиту, для ускорения затухания орбиты. 14 августа 2010 года он был выведен из эксплуатации и в 08:49 UTC 30 августа 2010 года спутник вошел в атмосферу.

\subsubsection{IceSat-2}

 Продолжателем миссии ICESAT стал спутник ICESAT-2. Этот спутник представляет собой систему для измерения высоты ледяного покрова и толщины морского льда, а также топографии суши, характеристик растительности и облаков. Входит в EOS. ICESAT-2 был запущен 15 сентября 2018 года с базы ВВС Ванденберг в Калифорнии ракетой Delta II на почти круговую околополярную орбиту с высотой примерно 496 км. Он был рассчитан на работу в течение трех лет, с запасом топлива на семь лет. Спутник вращается вокруг Земли со скоростью 6.9 километра в секунду. 
 
Миссия ICESAT-2 предназначена для получения данных о высоте, необходимых для определения баланса массы ледяного покрова, а также информации о растительном покрове. Он обеспечивает измерения топографии городов, озер и водохранилищ, океанов и поверхности суши по всему земному шару, в дополнение к полярному покрытию. Спутник также может определять топографию морского дна на глубине до 30 метров в чистых водах прибрежных районов. 
Единственным прибором на ICESAT-2 является усовершенствованная лазерная топографическая альтиметрическая система (ATLAS).  ATLAS излучает видимые лазерные импульсы длиной волны 532 нм. В процессе работы, система генерирует шесть лучей, связанные в три пары, чтобы лучше определять наклон поверхности и обеспечивать большее покрытие земли. Each beam pair is 3.3 km apart across the beam track, and each beam in a pair is separated by 2.5 km along the beam track. The laser array is rotated 2 degrees from the satellite's ground track so that a beam pair track is separated by about 90 m? . Частота лазерных импульсов в сочетании со скоростью спутника приводит к тому, что ATLAS замеряет высоту каждые 70 см вдоль наземного трека. 
Лазер срабатывает с частотой 10 кГц. Обратный импульс улавливается с помощью бериллиевого телескопа диаметром 79 см. Бериллий обладает высокой удельной прочностью и сохраняет форму в широком диапазоне температур. Телескоп собирает фотоны с длиной волны 532 нм, таким образом отфильтровывая посторонний свет в атмосфере. Компьютерные программы дополнительно фильтруют входящий поток, оставляя для анализа только отраженные фотоны лидара. 
Примечательным атрибутом ATLAS является то, что инженеры позволили спутнику контролировать его положение в космосе. Они сконструировали лазерную систему отсчета, которая подтверждает настройку лазера в соответствии с телескопом. 

Управляет научными данными ICESAT-2 Национальный центр данных по снегу и льду (The National Snow and Ice Data Center).

ICESAT-2 преследует четыре научные цели:
\begin{itemize}
    \item Количественно оценить вклад полярного ледяного покрова в текущее и недавнее изменение уровня моря и связь с климатическими условиями;
    \item Количественная оценка региональных признаков изменений ледникового покрова для оценки механизмов, вызывающих эти изменения и улучшения прогнозных моделей ледникового покрова. Это включает в себя количественную оценку региональной эволюции изменений ледникового покрова;

    \item Оценить толщу морского льда для изучения обмена энергией, массой и влажностью между льдом, океаном и атмосферой;

    \item Измерение высоты растительного покрова, как основу для оценки крупномасштабных изменений биомассы.;
\end{itemize}
Кроме того, ICESAT-2 выполняет измерения облаков и аэрозолей, высоты океанов, внутренних водоемов (водохранилищ и озер), городов/, а также движения земли после землетрясений или оползней. 

Запуск ICESAT-2 состоялся 15 сентября 2018 года в 15:02 UTC. Чтобы поддержать непрерывность данных между выводом из эксплуатации ICESAT и запуском ICESAT-2, воздушная операция NASA IceBridge использовала различные самолеты для сбора данных о полярной топографии и измерения толщины льда с помощью комплектных лазерных высотомеров, радаров и других систем. 

ICESAT-2 всё ещё находится на орбите и передает данные. 


\subsubsection{CryoSat-1 и CryoSat-2}
Другим проектом по изучению полярного льда является CryoSat. Это программа Европейского Космического Агентства (ESA) для мониторинга изменений протяженности и толщины полярного льда с помощью спутника на низкой околоземной орбите.  Предоставленная информация о поведении прибрежных льдов, будет иметь ключевое значение для более точных прогнозов изменения уровня моря. Аппарат CryoSat-1 был потерян в результате неудачного запуска в 2005 году, однако программа была возобновлена успешным запуском замены, CryoSat-2, запущенным 8 апреля 2010 года. 

CryoSat управляется Европейским центром космических операций (European Space Operations Centre -- ESOC) в Дармштадте, Германия.
Основным прибором CryoSat является SIRAL (SAR (synthetic-aperture radar?)  Interferometric radar altimeters/ интерферометрический радарный альтиметр). SIRAL работает в одном из трех режимов, в зависимости от того, где (над поверхностью Земли) летел CryoSat. Над океанами и ледяным покровом спутник работает как традиционный радарный высотомер. Над морским льдом когерентно передаваемые эхосигналы комбинируются (обработка синтетической апертуры) для уменьшения воздействия на поверхность, чтобы CryoSat мог отображаться более мелкие льдины. Самый продвинутый режим CryoSat используется вокруг границ ледяного покрова и над горными ледниками. Здесь альтиметр выполняет обработку синтетической апертуры и использует вторую антенну в качестве интерферометра для определения поперечного угла до самого раннего отраженного сигнала радара. Это обеспечивает точное местоположение измеряемой поверхности, когда поверхность наклонена. 

Хотя CryoSat-2 в значительной степени такой же, как и исходный спутник, в него был включен ряд ключевых улучшений. Самым значительным было решение поставить полностью дублированную полезную нагрузку, чтобы позволить продолжить миссию, если неисправность приведет к потере радара SIRAL. Другие изменения были вызваны устареванием первоначальной конструкции, некоторые из них повысили надежность, а другие упростили работу спутника. Несмотря на все изменения, миссия остается прежней, а характеристики с точки зрения измерительных возможностей и точности остаются прежними. CryoSat-2 был запущен 8 апреля 2010 года в 13:57 UTC с космодрома Байконур на ракете "Днепр".

Второй инструмент, DORIS (Doppler Orbit and Radio Positioning Integration by Satellite) используется для точного расчета орбиты космического корабля. Множество ретрорефлекторов также установлено на борту, что позволяет проводить измерения с земли для проверки орбитальных данных, предоставленных DORIS.

После запуска CryoSat-2 был выведен на низкую околоземную орбиту с перигеем 720 км, апогеем 732 км, наклоном 92 градуса и периодом обращения 99.2 минуты. При запуске он имел массу 750 кг, и, как ожидается, проработает не менее трех лет. 

\end{document}
